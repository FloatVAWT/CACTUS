%!TEX root = ../User Guide.tex
\chapter{Output description}
This section describes the output files written by CACTUS.

\section{Revolution-averaged integral rotor loads}
Revolution averaged performance data for each revolution are written to a comma delimited file appended with \texttt{\_RevData.csv}.

\begin{table}[!htbp]
\centering
\caption{Revolution-averaged output data.}
\label{tbl:output_vars_rev}
\begin{tabular}{p{0.3\textwidth}p{0.6\textwidth}}
\toprule
Variable name & Description \\ \midrule
Rev                    & Revolution number \\
Power Coeff. (-)       & Revolution average machine power coefficient \\
Tip Power Coeff. (-)   & Revolution average machine power coefficient normalized with $U_\textrm{tip}$ instead of $U_\infty$ \\
Torque Coeff. (-)      & Revolution average torque coefficient \\
Fx Coeff. (-)          & Revolution average $x$-component of force coefficient \\
Fy Coeff. (-)          & Revolution average $y$-component of force coefficient \\
Fz Coeff. (-)          & Revolution average $z$-component of force coefficient \\
Power (kW)             & Revolution average machine power \\
Torque (ft-lbs)        & Revolution average machine torque \\
\bottomrule
\end{tabular}
\end{table}

\section{Temporal blade integrated loads}
Performance data for each time step are written to a comma delimited file appended with \texttt{\_TimeData.csv}.

\begin{table}[!htbp]
\centering
\caption{Temporal blade integrated loads.}
\label{tbl:output_vars_time}
\begin{tabular}{p{0.3\textwidth}p{0.6\textwidth}}
\toprule
Variable name & Description \\ \midrule
Normalized Time (-)                   & Normalized simulation time, $t_N=t U_\infty/R$ \\
Theta (rad)                           & Turbine rotational phase angle \\
Rev                                   & Revolution number \\
Torque Coeff (-)                      & Torque coefficient \\
Power Coeff (-)                       & Power coefficient \\
Fx Coeff. (-)                         & $x$-component of force coefficient \\
Fy Coeff. (-)                         & $y$-component of force coefficient \\
Fz Coeff. (-)                         & $z$-component of force coefficient \\
Blade Fx Coeff (-)                    & Contribution to $x$-component of force coefficient from blade \\
Blade Fy Coeff (-)                    & Contribution to $y$-component of force coefficient from blade \\
Blade Fz Coeff (-)                    & Contribution to $z$-component of force coefficient from blade \\
Blade Torque Coeff (-)                & Contribution to torque coefficient from blade \\
Strut Fx Coeff (-)                    & Contribution to $x$-component of force coefficient from strut \\
Strut Fy Coeff (-)                    & Contribution to $y$-component of force coefficient from strut \\
Strut Fz Coeff (-)                    & Contribution to $z$-component of force coefficient from strut \\
Strut Torque Coeff (-)                & Contribution to torque coefficient from strut \\
\bottomrule
\end{tabular}
\end{table}

\section{Temporal blade element loads}
When \texttt{Output\_ELFlag} is set to 1 in the namelist input file, element loads data for each time step are written to a comma delimited file appended with \texttt{\_ElementData.csv}.

\begin{table}[!htbp]
\centering
\caption{Temporal blade element loads.}
\label{tbl:output_vars_blade}
\begin{tabular}{p{0.3\textwidth}p{0.6\textwidth}}
\toprule
Variable name & Description \\ \midrule
Normalized Time (-) & Normalized simulation time, $t_N=t U_\infty/R$ \\
Theta (rad)         & Turbine rotational phase angle \\
Blade               & Blade number \\
Element             & Element number \\
Rev                 & Revolution number \\
AOA25 (deg)         & Local flow angle of attack, defined at element quarter-chord location. \\
AOA50 (deg)         & Reference 50\% chord flow angle of attack. Different from AOA25 when element is rotating in the local spanwise direction. \\
AOA75 (deg)         & Reference 75\% chord flow angle of attack. Different from AOA25 when element is rotating in the local spanwise direction. \\
AdotNorm (-)        & Normalized AOA rate  $\dot{\alpha}_\textrm{norm} = \dot{\alpha} c / 2 U_\textrm{loc}$ \\ 
Re (-)              & Element Reynolds number based on element chord \\
Mach (-)            & Element Mach number \\
Ur (-)              & Local flow speed ratio with freestream, $U_r = U_\textrm{loc}/U_\infty$ \\ 
CL (-)              & Element lift coefficient, $C_L=L/{\frac{1}{2} \rho U_\textrm{loc}^2 A_E}$ \\
CD (-)              & Element drag coefficient, $C_D=D/{\frac{1}{2} \rho U_\textrm{loc}^2 A_E}$ \\
CM25 (-)            & Element pitching moment coefficient about the quarter-chord location, $C_{M,25}=L/{\frac{1}{2} \rho U_\textrm{loc}^2 A_E c}$ \\
CLCirc (-)          & Circulatory component of element lift coefficient, $C_{L,\textrm{circ}}={L_\textrm{circ}}/{\frac{1}{2} \rho U_\textrm{loc}^2}$ \\
CN (-)              & Element normal force coefficient, $C_N = {N}/{\frac{1}{2} \rho U_\textrm{loc}^2 A_E}$ \\ 
CT (-)              & Element tangential force coefficient, $C_T = {T}/{\frac{1}{2} \rho U_\textrm{loc}^2 A_E}$ \\
Fx (-)              & Contribution to $x$-component of force coefficient from element \\
Fy (-)              & Contribution to $y$-component of force coefficient from element \\
Fz (-)              & Contribution to $z$-component of force coefficient from element \\
te (-)              & Contribution to torque coefficient from element \\
\bottomrule
\end{tabular}
\end{table}

\section{Wall output}
When a wall calculation is being performed and WallOutFlag is set to 1 in the namelist input file, summary output data for the wall calculation is written to a comma delimited file appended with either \texttt{\_GPData.csv} for a ground plane calculation, or \texttt{\_FSData.csv} for a free surface calculation.

\subsection{Wall source panel data}
\begin{table}[!htbp]
\centering
\caption{Wall outputs.}
\label{tbl:output_wall}
\begin{tabular}{p{0.3\textwidth}p{0.6\textwidth}}
\toprule
Variable name & Description \\ \midrule
X/R (-)             & $x$-location of the panel center normalized by $R$ \\
Y/R (-)             & $y$-location of the panel center normalized by $R$ \\
Z/R (-)             & $z$-location of the panel center normalized by $R$ \\
SourceDens/Uinf (-) & Source density on the panel normalized by $U_\infty$ \\
\bottomrule
\end{tabular}
\end{table}

\subsection{Free surface data}
\begin{table}[!htbp]
\centering
\caption{Free surface outputs.}
\label{tbl:output_free_surface}
\begin{tabular}{p{0.3\textwidth}p{0.6\textwidth}}
\toprule
Variable name & Description \\ \midrule
X/R (-)    & $x$-location of the panel center normalized by $R$ \\
Y/R (-)    & $y$-location of the panel center normalized by $R$ \\
Z/R (-)    & $z$-location of the panel center normalized by $R$ \\
U/Uinf (-) & Wall tangential velocity (nominal freestream direction) normalized by $U_\infty$ \\
dH/R (-)   & Free surface height (above un-deflected height) normalized by $R$ \\
\bottomrule
\end{tabular}
\end{table}

\section{Field velocities}
If \texttt{WakeGridOutFlag = 1} the induced velocity field on a Cartesian grid is computed and written to a file. This output data is split into multiple files, each file containing the field data at a single timestep, since long simulations with a high resolution 3-D Cartesian grid would produce very large output files.

This computation can add considerable time to the simulation, since the induced velocity is calculated at every point in the specified Cartesian grid.

The output filenames take the following format:

\begin{lstlisting}
[case name]_WakeDefData_[timestep number].csv
\end{lstlisting}

\begin{table}[!htbp]
\centering
\caption{Field velocity output data.}
\label{tbl:output_cartesian_velocity}
\begin{tabular}{p{0.3\textwidth}p{0.6\textwidth}}
\toprule
Variable name & Description \\ \midrule
Normalized Time (-) & Normalized simulation time, $t_N=t U_\infty/R$ \\
x/R (-)             & $x$-location of the data point normalized by $R$ \\
y/R (-)             & $y$-location of the data point normalized by $R$ \\
z/R (-)             & $z$-location of the data point normalized by $R$ \\
U/Uinf (-)          & $x$-component of induced velocity              \\
V/Uinf (-)          & $y$-component of induced velocity              \\
W/Uinf (-)          & $z$-component of induced velocity              \\
Ufs/Uinf (-)        & $x$-component of free-stream velocity          \\
Vfs/Uinf (-)        & $y$-component of free-stream velocity          \\
Wfs/Uinf (-)        & $z$-component of free-stream velocity          \\
\bottomrule
\end{tabular}
\end{table}

\section{Vortex filament data}
If WakeElementOutFlag = 1 information about each vortex filament is written to a file. This output data is split into multiple files, each file containing the vortex filament data at a single timestep.
Note that the data output is specified at the endpoints of the vortex filaments, rather than at the centers of each filament.
The output filenames take the following format:

\begin{lstlisting}
[case name]_WakeData_[timestep number].csv
\end{lstlisting}

\begin{table}
\centering
\caption{Vortex filament output data.}
\label{tbl:output_vortex_filaments}
\begin{tabular}{p{0.3\textwidth}p{0.6\textwidth}}
\toprule
Variable name & Description \\ \midrule
Normalized Time (-) & Normalized simulation time, $t_N=t U_\infty/R$ \\
Node ID             & A unique ID given to each distinct filament node (useful for tracing a particle's path in time) \\
Origin Node         & ID of the element node from which this filament was generated \\
x/R (-)             & $x$-location of the data point normalized by $R$ \\
y/R (-)             & $y$-location of the data point normalized by $R$ \\
z/R (-)             & $z$-location of the data point normalized by $R$ \\
U/Uinf (-)          & $x$-component of induced velocity              \\
V/Uinf (-)          & $y$-component of induced velocity              \\
W/Uinf (-)          & $z$-component of induced velocity              \\
\bottomrule
\end{tabular}
\end{table}
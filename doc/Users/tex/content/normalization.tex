%!TEX root = ../User Guide.tex
\chapter{Normalization parameters}
Some definitions of parameters used in normalize CACTUS input and output parameters are given in \Cref{tbl:normalization_parameters}.

\begin{table}[!htbp]
\centering
\caption{Parameters used for non-dimensionalization.}
\label{tbl:normalization_parameters}
\begin{tabular}{p{0.20\textwidth}p{0.70\textwidth}}
\toprule
Variable          & Description \\
\midrule
$\rho$            & Fluid density \\
$U_\infty$        & Freestream fluid flow speed \\
$A_T$             & Turbine reference area. Typically this reference area is chosen to be the projected frontal area of the volume swept by the rotor. \\
$R$               & Turbine reference radius \\
$\omega$          & Turbine rotation rate \\
$U_\textrm{local}$& Fluid flow speed local to an element. \\
$U_\textrm{tip}$  & Turbine tip speed. Defined as $U_\textrm{tip} = \omega R$. \\
$c$               & Element chord length \\
$A_E$             & Element planform area \\
\bottomrule
\end{tabular}
\end{table}

In general, most parameters have been normalized at the machine scale. Unless otherwise noted in the input and output descriptions below, the force, torque, and power coefficients are normalized as:

$$ C_F = \frac{F}{\frac{1}{2} \rho U_\infty^2 A_T} $$
$$ C_T = \frac{T}{\frac{1}{2} \rho U_\infty^2 A_T R} $$
$$ C_P = \frac{P}{\frac{1}{2} \rho U_\infty^3 A_T} $$